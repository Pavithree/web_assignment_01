 \section*{Important Notes}


  \subsection*{Submission}

  \begin{itemize}
    \item Solutions have to be submitted to the SVN repository.
      Use the directory name \texttt{groupname/assignment\@assignmentnumber{}/}
      in your group's repository.
%     \item The answer sheet must have the screenshots and the code where ever asked. 		Additional python \texttt{.py} file needs to be also added in the repository. 
    \item The name of the group and the names of all participating students must
      be listed on each submission.
      
%     \item With the submission of your solution you confirm that you created the
%       solution independently as a group, especially without using other
%       intellectual contributions.
%       In other words, you submission should not be
%       \href{https://en.wikipedia.org/wiki/Plagiarism}{plagiarism}!
%       Should the case occur that the submissions of multiple groups are
%       identical, none of these groups will receive credit.
    \item Solution format: all solutions as \emph{one} PDF document.
      Programming code has to be submitted as Python code to the SVN repository.
      Upload \emph{all} \texttt{.py} files of your program!
      Use \texttt{UTF-8} as the file encoding.
      \emph{Other encodings will not be taken into account!}
    \item Check that your code compiles without errors.
    \item Make sure your code is formatted to be easy to read.
      \begin{itemize}
        \item Make sure you code  has consistent
          \href{https://en.wikipedia.org/wiki/Indent_style}{indentation}.
        \item Make sure you comment and document your code
          adequately in English.
        \item Choose consistent and intuitive names for your identifiers.
      \end{itemize}
    \item Do \emph{not} use any accents, spaces or special characters in your
      filenames.
  \end{itemize}

%   \clearpage
  \subsection*{Acknowledgment}
	
    This pdfLaTeX template was adapted by Jun Sun based on the LuaLaTeX version by Lukas Schmelzeisen. 

\subsection*{\LaTeX}
Use \texttt{pdflatex or LuaLatex combiler for  assignment\_X.tex}  to build your PDF.
